% Insert table with summary of projects
%% Table generated by Excel2LaTeX from sheet 'practitioners_projects'
%DIF < \begin{table}[t]
%DIF >  \begin{table}[t]
%  \centering
%  \caption{A summary of interviewee roles}
%     \begin{tabular}{l l}
%     \toprule
%     \textbf{Roles} & \textbf{Role Type} \\
%     \toprule
%     Chief Machine Learning Engineer & Technical \\
%     Chief Scientific Officer & Technical \\
%     Head of Natural Language Understanding & Technical \\
%     Machine Learning Engineer (Founder) & Technical \\
%     Solution Architect & Technical \\
%     Director & Manager \\
%     Chief Machine Learning Engineer & Technical \\
%     Data Science Manager & Manager \\
%     Innovation Architect  & Technical \\
%     Chief Architect  & Technical \\
%     Data Scientist & Technical \\
%     AI Specialist & Technical \\
%     Director of Consulting & Manager \\
%     Data Scientist & Technical \\
%     Data Scientist & Technical \\
%     Machine Learning Engineer & Technical \\
%     Computational Biologist & Technical \\
%     Data Scientist & Technical \\
%     Site Lead & Manager \\
%     AI Engineer & Technical \\
%     Chief Data Architect & Manager \\
%     Principle Data Scientist & Manager \\
%     Data Scientist & Technical \\
%     \hline
%     \end{tabular}%
%  \label{tab:practioners}%
%DIF < \end{table}%
%DIF >  \end{table}%

\begin{table*}[h!]
  \centering
  \caption{Summary of participants demographics}
    \begin{tabular}{lp{12cm} }
    \toprule
    \textbf{Type} & \textbf{Count} \\
    \toprule
    Role & Chief Machine Learning Engineer (2), Chief Scientific Officer, Head of Natural Language Understanding,  Machine Learning Engineer (Founder), Solution Architect, Director, Data Science Manager, Chief Architect, Data Scientist (5), AI Specialist, Director of Consulting, Machine Learning Engineer, Computational Biologist, Site Lead, AI Engineer, Chief Data Architect, Principle Data Scientist \\
    \midrule

    Org. Size & Small-Medium sized (5),  Large size (11)\\
    \midrule

    Org. Business Focus & Product (6), Services (4), Consultancy (4) , Product \& services (1), Public services (1)\\
    \midrule

    Sector & Healthcare (1), Banking \& Financial services (3), Public agency (1), Pharmaceutical (1), Gaming (1), Paper \& Forest (1), Real-estate (1), Electricity (1), Media (1), Engineering and service (1), Computer software (2)\\
    \midrule

    ML Type & Computer vision (5), Speech recognition (2), MLOps (4), Analytics (5), Natural Language Processing (2), Recommendation systems (1)\\
    \bottomrule

    \end{tabular}%
  \label{tab:practioners}%
\end{table*}%





An exploratory multiple-case study \cite{Runeson2008} was conducted between March and August 2021. 
%The research data was primarily collected through semi-structured interviews ~\cite{myers2007qualitative, Runeson2008}. 
The main research questions (RQ) include:
\begin{itemize}
    \item RQ1. What practices are applied in the development, deployment and maintenance of industrial ML-enabled software systems?
    \item RQ2. What tools are used to support the development, deployment and maintenance of industrial ML-enabled software systems?
   % \item RQ3. What challenges remain to be addressed in the development, deployment and maintenance of industrial ML-enabled software systems?
\end{itemize}
%involving two interviewers and one or multiple interviewees. 

\subsection{Research design and case selection}
The main goal of our study is to understand the state-of-practice of ML-enabled systems' development and toolchain within the Finnish context. In this study, a case (Table \ref{tab:data_source_storage_mlframeworks}) is an organization in Finland with experience in developing, deploying and maintaining ML-enabled software systems. The main criterion for case selection is that the ML-enabled software system needs to be operational in a production environment. %As such, our primary focus is to gather information on the whole system development life cycle from experiment to operations.

%Based on the study objective, at the initial phase, 
We first identified relevant practitioners from different organizations to take part in the study and adapted an interview guide used in an earlier study \cite{Lwakatare2019}. For the current study, we modified questions under the project background and characteristics section of the interview guide in order to only inquire about operational ML systems rather than practitioners' general experience in ML projects. This was done to exclude ML systems that are in experimental or prototyping stage because they are observed not only to employ immature practices and toolchains but also face different SE challenges \cite{Lwakatare2019}. In addition, we added new questions that asked about the infrastructure and tools used for the different activities of ML system development and operations. The identified practitioners (or their organizations) were primarily known to be working on ML solutions by the researchers and others were gathered from LinkedIn. %\footnote{https://www.linkedin.com/}  LinkedIn targets were identified based on their indication of holding a position that we considered to be related to ML. 

We reached out to 37 organizations via e-mail out of which 16 agreed to participate in the study. %The e-mails also detailed the study objectives, scope and data collection procedure. 
Generally, practitioners were free to choose whether (and who) to participate in the study, but researchers purposefully ensured that the organizations were varied in terms of sector and size. 

Interviewed practitioners had varying roles: Chief Machine Learning Engineer (2), Chief Scientific Officer, Head of Natural Language Understanding,  Machine Learning Engineer (Founder), Solution Architect(2), Director, Data Science Manager, Chief Architect, Data Scientist (5), AI Specialist, Director of Consulting, Machine Learning Engineer, Computational Biologist, AI Engineer, Chief Data Architect, Principle Data Scientist.
%The average length of experience in the indicated role is 2.2 years while the minimum and maximum are 4 months and 8.6 years respectively. 
%Following the description of the roles and organisational structure given by interviewees, we grouped the interviewees to hold either a technical or managerial role, from this perspective, seventeen (74\%) interviewees held technical roles while six (26\%) held management or leadership type of roles. One of the technical interviewees was also a founder of the company. 
Academically, ten (43\%) practitioners hold a PhD degree, ten (43\%) hold a Master's degree and two (9\%) hold a Bachelor's degree. %and one (4\%) practitioner had selectively taken courses. %but did not complete the studies
% A table to show practitioners' profile
%% Table generated by Excel2LaTeX from sheet 'practitioners_projects'
%DIF < \begin{table}[t]
%DIF >  \begin{table}[t]
%  \centering
%  \caption{A summary of interviewee roles}
%     \begin{tabular}{l l}
%     \toprule
%     \textbf{Roles} & \textbf{Role Type} \\
%     \toprule
%     Chief Machine Learning Engineer & Technical \\
%     Chief Scientific Officer & Technical \\
%     Head of Natural Language Understanding & Technical \\
%     Machine Learning Engineer (Founder) & Technical \\
%     Solution Architect & Technical \\
%     Director & Manager \\
%     Chief Machine Learning Engineer & Technical \\
%     Data Science Manager & Manager \\
%     Innovation Architect  & Technical \\
%     Chief Architect  & Technical \\
%     Data Scientist & Technical \\
%     AI Specialist & Technical \\
%     Director of Consulting & Manager \\
%     Data Scientist & Technical \\
%     Data Scientist & Technical \\
%     Machine Learning Engineer & Technical \\
%     Computational Biologist & Technical \\
%     Data Scientist & Technical \\
%     Site Lead & Manager \\
%     AI Engineer & Technical \\
%     Chief Data Architect & Manager \\
%     Principle Data Scientist & Manager \\
%     Data Scientist & Technical \\
%     \hline
%     \end{tabular}%
%  \label{tab:practioners}%
%DIF < \end{table}%
%DIF >  \end{table}%

\begin{table*}[h!]
  \centering
  \caption{Summary of participants demographics}
    \begin{tabular}{lp{12cm} }
    \toprule
    \textbf{Type} & \textbf{Count} \\
    \toprule
    Role & Chief Machine Learning Engineer (2), Chief Scientific Officer, Head of Natural Language Understanding,  Machine Learning Engineer (Founder), Solution Architect, Director, Data Science Manager, Chief Architect, Data Scientist (5), AI Specialist, Director of Consulting, Machine Learning Engineer, Computational Biologist, Site Lead, AI Engineer, Chief Data Architect, Principle Data Scientist \\
    \midrule

    Org. Size & Small-Medium sized (5),  Large size (11)\\
    \midrule

    Org. Business Focus & Product (6), Services (4), Consultancy (4) , Product \& services (1), Public services (1)\\
    \midrule

    Sector & Healthcare (1), Banking \& Financial services (3), Public agency (1), Pharmaceutical (1), Gaming (1), Paper \& Forest (1), Real-estate (1), Electricity (1), Media (1), Engineering and service (1), Computer software (2)\\
    \midrule

    ML Type & Computer vision (5), Speech recognition (2), MLOps (4), Analytics (5), Natural Language Processing (2), Recommendation systems (1)\\
    \bottomrule

    \end{tabular}%
  \label{tab:practioners}%
\end{table*}%




Ten cases are large organizations and five constitute small or medium-sized organizations with revenues below €50 million based on 2020 financial reports.
%Most of the organizations had a specific focus of either products (7 of 16) or services (4 of 16), and few were consultancies (3 of 16). As listed in Table~\ref{tab:practioners}, the types of ML solutions worked on included: computer vision (3), automatic speech recognition (ASR) (2), general ML operations (2), analytics platforms with ML features (4), data analysis (1), natural language processing (NLP) (2), data lake (1), and recommendation systems (1). 

\subsection{Data collection}

Research data was primarily collected through semi-structured interviews conducted by two researchers. A total of 23 practitioners from the 16 organizations were interviewed between 24$^{th}$ March and 27$^{th}$ April 2021. All interviews were conducted virtually due to COVID-19. Each interview session took on average 80 minutes. 

During the interview session, the researchers presented details of the study and requested consent to record the interview. One researcher asked the question outlined in the interview guide that contained five broad categories: data management, model training, model deployment, model monitoring and general challenges. The interview guide was not strictly followed to allow probing questions depending on the interviewees' responses and their experience with the topic. All recorded interviews were automatically transcribed using Otter.ai and errors in the transcriptions were manually corrected by the researchers.  %An example of a topic, question and sub-question from our interview guide is shown in Table~\ref{tab:sample_questions}.

% Table: Interview guide extract
% Interview guide extract
%\begin{table*}[t]
%  \centering
%  \caption{An example of a topic, main question and a follow up question}
%    \begin{tabular}{ll}
%    \toprule
%    \multicolumn{1}{l}{\textbf{Topic}} &
%      Project background and characteristics \\
%    Question &
%      What type of machine learning based system are you currently working on or consulting about? \\
%    Sub-question &
%How would you classify the area of ML/AI you are involved in? for example NLP, Speech, Computer vision etc.\\
%    \hline
%    \end{tabular}%
%  \label{tab:sample_questions}%
%\end{table*}%


\subsection{Data analysis}
Analysis of the interview transcripts mainly consisted of two coding steps and a session to discuss and harmonize the codes \cite{Runeson2008}. A deductive approach formed the first stage of our coding process in which main themes informed by the structure of our interviews were outlined. %This facilitated a unified structure and terminology in the analysis. 
The themes were constituted high-level codes organized into six groups: role and responsibility, organisation, ML usecase, Practices, Challenges and Tools. The actual coding of data was done in an iterative manner using both deductive and inductive \cite{Runeson2008} approaches within each group and applied broadly at a paragraph or statement level. The sub-groups were further refined during researcher meetings. 
%Our study had one interviewee per organization and neither did we prescribe the type of roles to be interviewed. As a result, we observe that interviewees had deeper insight into subject matters that fall within their current role especially in larger organizations characterised by dedicated teams in different aspects of the pipeline. In smaller case companies, interviewees had a broader perspective of the entire pipeline activities as team meetings and high-level development decisions often involved everyone in the organization. This may affect generalizability of our results.
