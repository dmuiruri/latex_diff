
An exploratory multiple-case study \cite{Runeson2008} was conducted between March and August 2021. The research method was selected to gain a deep understanding of the enacted practices and tool support for ML systems in real-world settings. The main research questions (RQ) include:
\begin{itemize}
    \item RQ1. What practices are applied in the development, deployment, and maintenance of industrial ML-enabled software systems?
    \item RQ2. What tools are used to support the development, deployment, and maintenance of industrial ML-enabled software systems?
\end{itemize}

\subsection{Research design and case selection}
The main goal of our study is to understand the state-of-practice of ML-enabled systems' development and toolchain within the Finnish context. In this study, a case (Table  \ref{tab:data_source_storage_mlframeworks_interviewees}) is an organization in Finland with experience in developing, deploying, and maintaining ML-enabled software systems. The main criterion for case selection is that the ML-enabled software system must be operational in a production environment.

We first identified relevant practitioners from different organizations and adapted an interview guide used in earlier research \cite{Lwakatare2019}. For the current study, we modified questions under the project background and characteristics section of the interview guide to only inquire about operational ML systems rather than practitioners' general experience in ML projects. This was done to exclude ML systems in the experimental stage because they often have immature practices and toolchains \cite{Lwakatare2019}. In addition, we added new questions to the interview guide that inquired about the infrastructure and tools used. The identified practitioners (or their organizations) were primarily known to be working on ML solutions by the researchers, and others were gathered from LinkedIn.

We reached out to 37 organizations via e-mail, out of which 16 agreed to participate in the study. Generally, practitioners were free to choose whether (and who) to participate, but researchers purposefully ensured that the organizations were varied in terms of sector and size. Interviewed practitioners had varying roles, as seen in Table  \ref{tab:data_source_storage_mlframeworks_interviewees}. Academically, ten (43\%) practitioners hold a Ph.D. degree, ten (43\%) hold a Master's degree, and two (9\%) have a Bachelor's degree. Ten cases are large organizations, and five constitute small or medium-sized organizations with revenues below €50 million based on 2020 financial reports.


\subsection{Data collection}
Research data was primarily collected through semi-structured interviews conducted by two researchers. A total of 23 practitioners from the 16 organizations were interviewed between 24$^{th}$ March and 27$^{th}$ April 2021. All interviews were conducted virtually due to COVID-19. Each interview session took, on average, 80 minutes. 

During the interview session, the researchers presented details of the study and requested consent to record the interview. One researcher asked the question outlined in the interview guide that contained five broad categories: data management, model training, model deployment, model monitoring, and general challenges. The interview guide was not strictly followed to allow probing questions depending on the interviewees' responses and expertise. All recorded interviews were automatically transcribed using Otter.ai, and the researchers manually corrected errors in the transcriptions.  

% Interview guide extract
%\begin{table*}[t]
%  \centering
%  \caption{An example of a topic, main question and a follow up question}
%    \begin{tabular}{ll}
%    \toprule
%    \multicolumn{1}{l}{\textbf{Topic}} &
%      Project background and characteristics \\
%    Question &
%      What type of machine learning based system are you currently working on or consulting about? \\
%    Sub-question &
%How would you classify the area of ML/AI you are involved in? for example NLP, Speech, Computer vision etc.\\
%    \hline
%    \end{tabular}%
%  \label{tab:sample_questions}%
%\end{table*}%


\subsection{Data analysis}
Analysis of the interview transcripts mainly consisted of two coding steps and a session to discuss and harmonize the codes \cite{Runeson2008}. A deductive approach formed the first stage of our coding process in which main themes informed by the structure of our interviews were outlined. The themes constituted the high-level codes in our analysis, and these included: role and responsibility, organization, ML usecase, Practices, Challenges, and Tools. The actual coding of data was done in an iterative manner using both deductive and inductive \cite{Runeson2008} approaches within each group and applied broadly at a paragraph or statement level. The sub-groups were further refined during researcher meetings.

