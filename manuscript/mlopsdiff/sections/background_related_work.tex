
\subsection{Software engineering (SE) for ML}
Adaptation and incorporation of well-established SE methods in the development of ML systems are crucial \cite{Amershi2019}because they emphasize other important \DIFaddbegin \DIFadd{engineering }\DIFaddend aspects beyond ML algorithms \cite{Sculley2015}. With the practices, organizations can address several challenges reported at the different stages of the taxonomy that \DIFdelbegin \DIFdel{depicts the evolution of the use }\DIFdelend \DIFaddbegin \DIFadd{depict evolution }\DIFaddend of ML components in software-intensive systems (experimentation, non-critical deployment, critical deployment, cascading deployment, and autonomous ML components) \cite{Lwakatare2019}.

Serban and van der Blom \cite{Serban2020Practices} developed a catalog of 29 SE practices for ML applications based on literature and later measured their adoption rate through a survey with 313 practitioners. The catalog includes SE practices about data (e.g., employing sanity checks for all external data sources), training (e.g., use versioning for data, model, and training scripts), coding (e.g., using continuous integration), deployment (e.g., automate model deployment), team (e.g., collaborating with multidisciplinary team members), and governance (enforcing fairness and privacy). Compared to our study, some of the reported SE practices are too general. The lack of details gives room to multiple interpretations, for example, the named practice to use continuous integration. In addition, we study practice enactment also by investigating the toolchains, for which the authors \cite{Serban2020Practices} had only speculated to influence the adoption rate of specific practices. 

\subsection{ML workflow and pipelines}

ML workflows describe different tasks performed to develop, deploy, and operate ML models  \cite{Amershi2019}. ML pipelines express complex input/output relationships between different tasks/operators of an automated ML workflow \cite{Doris2021MLPipelines}. Generally, ML pipelines plug together several tools to automate ML workflows \cite{Hummer2019IBM}.

A typical ML workflow life cycle includes model requirements, data collection, cleaning, labeling, feature engineering, model training, evaluation, deployment, and monitoring  \cite{Amershi2019}. Studies show that end-to-end automation of ML workflows improves ML models’ quality, traceability, development time, and deployment rate  \cite{Doris2021MLPipelines, Hummer2019IBM}. Furthermore, it allows organizations to reuse common workflow steps across multiple ML systems \cite{Baylor2017, Hummer2019IBM}.

Few studies report the characteristics of ML pipelines in terms of their components, architectures, and tools \cite{Hummer2019IBM,Doris2021MLPipelines}. Unlike our qualitative analysis, Xin et al \cite{Doris2021MLPipelines} quantitatively analyzed over 3000 ML pipelines at Google and presented their high-level characteristic concerning pipeline lifespan, complexity, and resource consumption. For the complexity of ML pipelines, the authors analyzed typical input data shape, feature transformation, and model diversity. Model diversity showed that a large portion was neural networks (NN) (64\%) and model type and architecture influence the characteristics of the resulting ML pipelines. The authors \cite{Doris2021MLPipelines} identified data management-related areas as key for optimizing ML pipelines.