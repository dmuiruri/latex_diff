\subsection{Model training}

\underline{\emph{ML algorithm selection and transfer learning}}
\DIFdelbegin \DIFdel{are commonly occurring practices. Selection }\DIFdelend \DIFaddbegin \DIFadd{The choice }\DIFaddend of ML algorithms is \DIFdelbegin \DIFdel{largely }\DIFdelend influenced by training data type and formulation of the learning problem during requirement elicitation. Heuristics are used in cases H and I to complement ML algorithms\DIFdelbegin \DIFdel{, }\DIFdelend \DIFaddbegin \DIFadd{; }\DIFaddend in both cases, an explicable decision based on \DIFdelbegin \DIFdel{heuristic algorithms is highly regarded }\DIFdelend \DIFaddbegin \DIFadd{heuristics is preferred }\DIFaddend compared to an ML solution with high \DIFdelbegin \DIFdel{accuracy but largely }\DIFdelend \DIFaddbegin \DIFadd{prediction accuracy but }\DIFaddend inexplicable. The \DIFdelbegin \DIFdel{trade-offs arise either due to regulatory constraints or where a heuristic based approach provides a much simpler solution compared to a complex ML with a closely similar result}\DIFdelend \DIFaddbegin \DIFadd{ML-heuristic trade-off tends to arise due to business sector regulatory constraints}\DIFaddend .

%DIF < A similar learning problem can be approached in different ways, for example the problem of extracting information from documents addressed in Cases B and C. A pre-study conducted in Case B showed an NLP approach not to be fit for their problem instead two supervised ML models are used. One model is based on convolutional variational layers using a Bayesian approach and mostly handles non-machine readable documents with an image of the document as input. The second model is a convolutional pyramid model based on convolutional neural networks (CNN) layers used mostly for machine readable invoices with a chargrid compressed image as input. A clustering algorithm is also used in the second model to automatically group the documents into similarity groups e.g., documents from different suppliers. Case C implements classifiers based on CNN to recognize the types of document images.
\DIFdelbegin %DIFDELCMD < 

%DIFDELCMD < %%%
\DIFdelend Transfer learning is \DIFdelbegin \DIFdel{indicated as the main approach to train NN efficiently since model parameters can take a long time to converge and require }\DIFdelend \DIFaddbegin \DIFadd{typically used to train large NN efficiently, for example, in speech recognition and computer vision settings. This is mainly because model convergence can be a prolonged process that requires }\DIFaddend significant computing resources. Transfer learning is based \DIFdelbegin \DIFdel{either }\DIFdelend on publicly available models or proprietary models.

\DIFdelbegin \DIFdel{Computer vision systems in }\DIFdelend \DIFaddbegin \DIFadd{In }\DIFaddend cases A and F\DIFdelbegin \DIFdel{make use of transfer learning by applying state of the art models available on a wide range of }\DIFdelend \DIFaddbegin \DIFadd{, computer vision systems utilize transfer learning to test different }\DIFaddend CNN architectures. Case M's NLP solution \DIFdelbegin \DIFdel{was also }\DIFdelend \DIFaddbegin \DIFadd{is }\DIFaddend trained using transfer learning \DIFdelbegin \DIFdel{mainly }\DIFdelend to overcome data insufficiency challenges. Case B applies transfer learning based on proprietary models as a cost management strategy. 
%DIF < The setup consists of an ensemble of six models and each model is trained for 15 epochs. If the training data set is too large, one epoch can take up to 5 hours, meaning a total of 450 hours computing hours to train the entire ensemble. When transfer learning is applied, only the last 4 epochs are trained which provides significant cost reduction. Essentially, case B maintains two training cycles, a low frequency training of large models and higher frequency training cycle based on transfer learning.

Training NN without \DIFdelbegin \DIFdel{using }\DIFdelend transfer learning can \DIFdelbegin \DIFdel{also be motivated by several factors we observe in case }\DIFdelend \DIFaddbegin \DIFadd{be driven by two factors observed in cases }\DIFaddend D and E. \DIFdelbegin \DIFdel{(1) The amount of data is considered sufficient }\DIFdelend \DIFaddbegin \DIFadd{One, there is sufficient data and computing resources }\DIFaddend for training a model to convergence. \DIFdelbegin \DIFdel{(2) Availability of required computing resources. (3) Limited availability of relevant }\DIFdelend \DIFaddbegin \DIFadd{Two, there is limited availability of suitable }\DIFaddend open-source models\DIFdelbegin \DIFdel{in a domain.
%DIF < Case D and E projects train the ASR models with their own data. Research grants or credits provided by their cloud providers facilitated access to computing infrastructure in the early phase of the project. A full model training cycle in case E can relatively take two weeks, for example Case D's training data had low resource for Finnish languages within the healthcare sector.
}\DIFdelend \DIFaddbegin \DIFadd{.
}\DIFaddend 

\underline{\emph{ML frameworks}}
used across the cases can be broadly categorized as either Neural Network (NN) or classical (non-NN) \DIFdelbegin \DIFdel{ML solutions. Tensorflow (https://bit.ly/38sgWc4) and PyTorch (https://bit.ly/3gMHnxG) }\DIFdelend \DIFaddbegin \DIFadd{frameworks. Tensorflow-Keras and PyTorch }\DIFaddend are the two commonly used frameworks for developing \DIFdelbegin \DIFdel{DL modelsas summarised }\DIFdelend \DIFaddbegin \DIFadd{NN models, as summarized }\DIFaddend in  Table~\DIFdelbegin \DIFdel{\ref{tab:data_source_storage_mlframeworks}.
Practitioners who used TensorFlow tended to make use of the Keras (https://keras.io/) framework which abstracts the low-level syntax found in the native TensorFlow framework.
}\DIFdelend \DIFaddbegin \DIFadd{\ref{tab:data_source_storage_mlframeworks_interviewees}.
}\DIFaddend 

Although \DIFdelbegin \DIFdel{NN frameworks }\DIFdelend \DIFaddbegin \DIFadd{ML frameworks may }\DIFaddend provide similar core features, \DIFdelbegin \DIFdel{a few factors can affect }\DIFdelend the choice of \DIFdelbegin \DIFdel{framework. (1) }\DIFdelend \DIFaddbegin \DIFadd{the framework can be based on }\DIFaddend a framework's usability, \DIFdelbegin \DIFdel{(2) a framework's }\DIFdelend \DIFaddbegin \DIFadd{flexibility, or }\DIFaddend underlying efficiency in utilizing computing resources\DIFdelbegin \DIFdel{, (3) a framework's flexibility. A case in point, }\DIFdelend \DIFaddbegin \DIFadd{. For example, both }\DIFaddend cases D and E develop \DIFdelbegin \DIFdel{an ASR solutions but make use of }\DIFdelend \DIFaddbegin \DIFadd{ASR models but use }\DIFaddend Kaldi and PyTorch frameworks\DIFdelbegin \DIFdel{respectively. %DIF < Similarly, NLP models in cases M and N were developed using PyTorch and Watson (https://ibm.co/3BrGa6P) respectively. 
}\DIFdelend \DIFaddbegin \DIFadd{, respectively. 
}\DIFaddend Frameworks can mature into \DIFdelbegin \DIFdel{certain domains much later }\DIFdelend \DIFaddbegin \DIFadd{specific domains at varying rates, }\DIFaddend and therefore teams might \DIFdelbegin \DIFdel{seemingly use different frameworks out of }\DIFdelend \DIFaddbegin \DIFadd{adopt different frameworks for }\DIFaddend such historical reasons. %DIF < In non-DL setups, Scikit-Learn (https://bit.ly/3t2jyqH) and XGBoost (https://bit.ly/3t4lKhw) were dominant frameworks/libraries used to implement ML solutions.
\DIFdelbegin \DIFdel{Specialised analytics }\DIFdelend \DIFaddbegin \DIFadd{Analytics }\DIFaddend frameworks such as Spark \DIFdelbegin \DIFdel{(https://bit.ly/3gLVtQ7) }\DIFdelend also feature in case I. 
\DIFdelbegin \DIFdel{We generally note that team members freely adopt frameworks suitable for accomplishing tasks efficiently. %DIF < This was however best supported in two scenarios: (1) Teams with a pre-defined data type abstraction as discussed in the previous data storage formats section, accompanied by a dedicated team to manage deployment operations e.g., in case G. (2) When modelling results do not require elaborate deployment techniques, often the end result is a report or exploratory analysis. 
%DIF < For instance, Case N indicated one team member's preference to use R (https://bit.ly/3zwVoqK) and RStudio (https://bit.ly/38s1hcS) in generating reports and analysis while the rest of the team mainly uses Python and Python based frameworks.
%DIF <  Ad-hoc experiments
%DIF <  An alternative view pertaining to model training relates to whether the setup is an IoT or a non-IoT setting. IoT settings often mean that data is streamed from multiple devices or sensors and setups in cases A, I, and J were observed to l
}\DIFdelend 

%DIF < The ML problem of extracting information from documents, is addressed in Cases B and C using different approaches. A pre-study conducted in Case B showed an NLP approach not to be fit for their problem instead two supervised ML models are used. One model is based on variational convolutional of layers using a Bayesian approach and mostly handles non-machine readable invoices. The model takes as input an image of an invoice and outputs the extracted fields from the image. Specifically, an OCR is used to extract characters from the image and these are combined and fed into the model so as to get the recognized values. The second model is a convolutional pyramid model based on CNN layers used mostly for machine readable invoices. The second model takes as input a chargrid compressed image of an invoice and outputs a heatmap that highlights areas of interest for text extraction. A clustering algorithm is also used in the second model to automatically group the invoices into similarity groups e.g., invoices from different suppliers. On the other hands, Case C implements classifiers based on CNN to recognize the types of document images 
%DIF <  Case C: for the fields’ recognition side, so we basically try to identify which document arrives to us. And we were testing convolutional neural networks, and see how these convolutional neural networks, and text analysis, and combination of these, the accuracy is something like fully around 93/ 94%. And once we know what document we have at hand, then we need to see the kind of template to, because we are getting a lot of pictures taken with a smartphone and they have skew errors,  perspective errors and we're trying to fix them with keypoint descriptors, with image alignment, basically. And once we know that, and we know in which x and y coordinates we have relevant fields in the templates that we can OCR the specific coordinates. A bit unstructured data a bit more structured
%DIF <  For Case B, models are trained using Tensorflow and one training cycle is for six models trained from the scratch. For re-training transfer learning is used to train pre-trained models on new data.
\DIFdelbegin %DIFDELCMD < 

%DIFDELCMD < %%%
\DIFdelend Overall, challenges in model training relate to infrastructure costs, complexities of tuning\DIFaddbegin \DIFadd{, }\DIFaddend and identifying explainable factors about a model's performance.

%DIF >  Summary table for practices and challenges
\DIFaddbegin % Table summary of practices in ML flow
\begin{table*}[t]
  \caption{Summary of practices and challenges}
  \centering
  \resizebox{15cm}{8cm}{%
  \begin{tabular}{p{1,8cm}p{9cm}p{8cm}}
    \toprule
     & \textbf{Practices} &\textbf{Challenges}  \\
    \toprule 
    \multicolumn{2}{l}{ \textbf{ML workflow}} \\ 
    % Data Management
    \multirow{4}{*}{ Data management } & 
    \begin{itemize}
       \item Batch or stream data loads largely from internal systems, third party vendors or devices and sensors
       \item Co-location of data and compute to reduce I/O latency in data transfer
       \item Selecting data storage formats (e.g., Apache Parquet) with great consideration of scalability, portability, ML frameworks
       \item Data documentation (e.g., data catalogue) for fast data identification
       \item Employing data validation approaches (e.g., descriptive statistics and schema) that are tailored to the types of data 
       \item Maintaining data quality by a dedicated team or third party vendor
       \item Determining data quality metrics from domain knowledge especially in highly specialized settings
    \end{itemize}
    &
    \begin{itemize}
       %\item Organizations contend between using public cloud infrastructure or private infrastructure due to data security and residency constraints.
       \item Determining ownership of data quality aspects especially in large organizations or when data collection is outsourced
       \item IoT related factors such as sensor outage, network latency or low traffic priority, sensor quality etc.
       \item Programming defaults in data collection components can lead to poor data quality through subtle hard to notice errors.
       \item Lack of standardized annotation formats across DL networks especially in computer vision reduces interoperability across network architectures.
    \end{itemize} \\

    % Model Training
    \midrule
    \multirow{3}{1,8cm}{Model training \& evaluation} & 
    \begin{itemize}
       \item Selecting ML algorithm based on available data and learning problem formulation during requirement elicitation and exploratory experiments
       \item Using heuristics to compliment or over ML algorithm when constrained by regulations or the complexity of models
       \item Employing transfer learning to effectively and accurately train DL models.
       \item Flexibility to choose standard ML frameworks e.g. Tensorflow and PyTorch as popular in DL, and Scikit-Learn and XGBoost in non-DL
       \item Using ML frameworks that offer great flexibility, efficiency and usability 
       \item Employing multiple approaches to evaluate quality of ML models e.g., using validation dataset stratified by quality
       \item Managing and tracking model evaluation results using experiment tracking tools, or metadata and hash-based approaches.
    \end{itemize}
    &
    \begin{itemize}
       \item The cost of training deep learning models from a clean start can be prohibitively high
       %\item In some IoT cases, different data regimes can emerge from a sensor's contextual environment. In some cases, such an outcome may mean training multiple models since deploying a single model may result in poor inference.
       \item Determining model explainability
       \item Feature extraction and hyper parameter tuning can be a time consuming activity especially in organization with different types of data.
       \item Model benchmarking was highlighted as an inherently difficult task given that it is challenging to replicate publicly available state of the art models and related results.
    \end{itemize} \\
     \midrule
    % Monitoring
    \multirow{3}{1,8cm}{Model Deployment \& Monitoring} & 
    \begin{itemize}
       \item Inference serving through REST based API endpoints deployed in public cloud environments
       \item Inference serving with strict latency requirements through gRPC endpoints as opposed to REST endpoints.
       \item Model deployment for either batch inference or online inference purposes
       \item Monitoring at different parts of the pipeline, to ensure data quality, model quality and performance and infrastructure utilization
    \end{itemize}
    &
    \begin{itemize}
       \item Deploying models within organizations that do not use the cloud environment can be a lengthy process due to relevant data governance protocols.
       \item Monitoring model or data drift in deployed systems can be a challenge due to lack of visibility especially in scenarios where input data cannot be saved due to GDPR related constraints.
    \end{itemize}
    \\
        % Tools and infrastructure in ML Pipelines
    \midrule
    \multirow{3}{*}{\textbf{ML Pipeline}} & 
    \begin{itemize}
       \item Version control code and all pipeline related artifacts e.g. in git, and provision execution environment using infrastructure-as-code frameworks e.g. Terraform
       \item Encapsulating ML training workflows in docker containers to increase portability 
       \item Using common container orchestration platforms e.g., Kubernetes to build scalable containerised pipelines
       \item Using ML workflow automation tools e.g. Argo and kubeflow to execute schedule ML training pipelines and queues
       \item Tracking ML training experiments largely in custom ways e.g. hashing and custom web tools but also with ML workflow automation tools.
       \item Employing continuous integration tools e.g., Jenkins to test and build docker images prior to deployment
    \end{itemize}
    &
    \begin{itemize}
       \item Maintaining an up to date stack of tools frameworks requires rigorous testing to avoid regression errors and dependency breaks across tool chains.
       \item Pipelines can become quite complex especially when dealing with complex DL architectures where multiple models are maintained.
       \item Skills required to run end-to-end automated ML pipelines are not easily available.
    \end{itemize}
    \\

    \midrule

    \end{tabular}%
    }
  \label{tab:practices_challenges}%
\end{table*}%


\DIFaddend \subsection{Model evaluation and experiment management}
Model training is \DIFdelbegin \DIFdel{considerably }\DIFdelend an iterative process \DIFdelbegin \DIFdel{involving (1) determining }\DIFdelend \DIFaddbegin \DIFadd{with distinct stages; determining the }\DIFaddend suitability of data and algorithms, \DIFdelbegin \DIFdel{(2) parameter and hyperparameter optimization}\DIFdelend \DIFaddbegin \DIFadd{parameter }\DIFaddend and \DIFdelbegin \DIFdel{(3) }\DIFdelend \DIFaddbegin \DIFadd{hyper-parameter optimization, and }\DIFaddend model evaluation. %DIF < These iterations result in multiple model variants with their respective attributes, such as accuracy and parameter settings, which calls for 
Managing metadata from these \DIFdelbegin \DIFdel{experiments makes the training process }\DIFdelend \DIFaddbegin \DIFadd{stages makes the ML workflow }\DIFaddend traceable and reproducible.

We note three unique approaches used to evaluate models\DIFdelbegin \DIFdel{: (1) Data is stored such that it can be stratified by qualityallowing composition of }\DIFdelend \DIFaddbegin \DIFadd{. One, data is stored according to its quality, which enables composing datasets with different levels of quality for }\DIFaddend training and validation \DIFdelbegin \DIFdel{data to include different quality }\DIFdelend \DIFaddbegin \DIFadd{purposes }\DIFaddend (Cases D and E)\DIFdelbegin \DIFdel{, (2) Use of ensemble of models each }\DIFdelend \DIFaddbegin \DIFadd{. The second approach uses model ensembles, where each model is }\DIFaddend trained on a unique subset of the data (Case B)\DIFdelbegin \DIFdel{and (3) Use of }\DIFdelend \DIFaddbegin \DIFadd{. The third approach applies }\DIFaddend a configurable inference algorithm where each configuration \DIFdelbegin \DIFdel{makes use of }\DIFdelend \DIFaddbegin \DIFadd{uses }\DIFaddend a unique adaptation of the model (Case E). 
%DIF < In the third setup, a model is composed of a core part supporting feature sharing and different adaption layers tuned for specific use cases. Each configuration of the model has an associated test data used to validate the model. This was encountered in case E where 30 to 40 configurations of the model were maintained.

To manage model evaluation results\DIFdelbegin \DIFdel{from these kinds of setups}\DIFdelend , case organizations either use dedicated experiment tracking tools \DIFdelbegin \DIFdel{case (}\DIFdelend \DIFaddbegin \DIFadd{(case }\DIFaddend G, I, N, O\DIFaddbegin \DIFadd{, }\DIFaddend and P), \DIFdelbegin \DIFdel{logging }\DIFdelend \DIFaddbegin \DIFadd{log }\DIFaddend process metadata (case B, E, F) or \DIFdelbegin \DIFdel{generating }\DIFdelend \DIFaddbegin \DIFadd{generate }\DIFaddend hashes (case D). \DIFdelbegin \DIFdel{These approaches are summarised in Table~\ref{tab:databases}. 
}\DIFdelend \DIFaddbegin \DIFadd{Hashing involves computing hash values on given combinations of ML artifacts (data, configurations, model) following the execution of an ML pipeline.
}\DIFaddend 


\DIFdelbegin \DIFdel{Case D uses hashing such that a hash is computed from a given version of the datacombined with a model's parameters.
The resulting hash is stored for later reference. Obtaining a previously stored hash implies that a model if similar characteristics already exists. %DIF < based on that data version and parameter settings has been previously trained and the model can be identified by a given version number. 
}%DIFDELCMD < 

%DIFDELCMD < %%%
\DIFdelend Case E and F \DIFdelbegin \DIFdel{utilize the generation and collection of metadata which includes metadata collected from tools such as git hashes.Case E stores metadata in a data warehouse which can be queried to produce spreadsheets reports. Case F's platform generates metadata at each step of the pipeline and resulting data is visualized on a web tool.
}\DIFdelend \DIFaddbegin \DIFadd{generate and collect metadata (e.g., Git hashes), which are used to produce custom reports. These approaches are summarised in Table~\ref{tab:databases}.
}

\DIFaddend Systematic management of experiments facilitates workflow automation \DIFdelbegin \DIFdel{. %DIF < The same metadata can be utilised by the MLOps platform to re-run a model using a previous configuration.
}\DIFdelend \DIFaddbegin \DIFadd{and further increases the traceability and reproducibility of ML workflows.
}\DIFaddend 

 %DIF < Case P tracks experiments using two approaches. One approach makes use of the MLflow (https://bit.ly/38sttfK) tool. The second approach allows data scientists to freely generate metrics in JSON format and store them in DynamoDB (https://amzn.to/3mMHGwh). The second approach facilitates storage of metrics without following any strict schema being enforced because of the NoSQL nature of the DynamoDB. 
\DIFdelbegin %DIFDELCMD < 

%DIFDELCMD < %%%
%DIF < In the overall, we observed that such systematic management of experiments facilitates workflow automation since data required to repeat a tracked stage of the pipeline is readily available and the transparency of the entire pipeline increased. 
 \DIFdelend