% Summary context, data collection, storage platforms
%% % Table summary of data collection, storage and ML frameworks
%% \begin{table}[t]
%%   \centering
%%   \caption{Summary of data sources and storage platforms}
%%   \begin{tabular}{clll}
%%     \hline
%%     Case & Data Source & Storage Solution & ML-Framework & ML-Category \\
%%     \hline
%%     A & Camera & Google cloud & Tensorflow & DL \\
%%     B & End users & AWS Tensorflow & DL \\
%%     C & End users & On-premise & Tensorflow, PyTorch, OpenCV & DL \\
%%     D & End users & Google cloud & Kaldi ASR framework  & DL \\
%%     E & Open source, End users & Google cloud & PyTorch & DL \\
%%     F & Camera & - & Tensorflow (Keras) and other frameworks & DL \\
%%     G & Trading systems & AWS & Tensorflow, Scikit-Learn & DL and Non-DL \\
%%     H & End users and internal systems & AWS & Scikit-Learn, XGBoost, Heuristics & Non-DL \\
%%     I & Sensor  and Technicians & AWS & Spark Analytics, Heuristics/Rules & Non-DL \\
%%     J & Sensor & Azure Cloud & - & Non-DL \\
%%     K & Meters & Azure Cloud & Scikit-learn, XGBoost & Non-DL \\
%%     L & Medical Device,  Genome Sequence & Azure Cloud & R & Non-DL \\
%%     M & System users & Azure Cloud & PyTorch, Scikit-learn (Classification) & Non-DL \\
%%     N & System users & AWS & Watson(IBM), Tensorflow, R & DL \\
%%     O & System users & Azure Cloud & - & Non-DL \\
%%     P & System users & AWS & Scikit-learn, Tensorflow(Keras), fastText & Non-DL \\
%%     \hline
%%     \end{tabular}%
%%   \label{tab:data_source_storage_mlframeworks}%
%% \end{table}%


% Table summary of data collection, storage and ML frameworks
%DIF > \begin{table*}[t]
%DIF >   \centering
%DIF >   \caption{Summary of ML usecase, frameworks, data sources and storage platforms across cases. (*of the ML usecase)}
%DIF >   \begin{tabular}{p{0,5cm}p{3,1cm}p{2cm}p{2,7cm}p{1,6cm}p{2,7cm}}
%DIF >     \toprule
%DIF >     \textbf{Case} & \textbf{ML usecase (Type)} & \textbf{Domain*} &\textbf{Data Source} & \textbf{Storage} & \textbf{ML-Framework} \\
%DIF >     \toprule \\
%DIF >     A & Object Detection (NN) & IoT & Camera & Google Cloud & Tensorflow \\
%DIF >     B & Form data extraction  (NN) & Finance & Internal Systems & AWS & Tensorflow \\
%DIF >     C & Form data extraction (NN) & Public Services & Internal Systems & On-premise & Tensorflow, PyTorch \\
%DIF >     D & Transcription (NN) & Healthcare & Internal Systems & Google Cloud & Kaldi ASR framework \\
%DIF >     E & Speech based UI (NN) & E-Commerce & Open-source, End-users & Google Cloud & PyTorch \\
%DIF >     F & MLOps (NN, Non-NN) & IT Services & Camera  & - & Multiple frameworks \\
%DIF >     G & MLOps (NN, Non-NN)  & Energy & Internal systems & AWS & Tensorflow, Scikit-Learn \\
%DIF >     H & Risk management (Non-NN) & Finance & Internal systems & AWS & Scikit-Learn, Heuristics \\
%DIF >     I & Predictive maintenance (Non-NN) & Engineering & Sensor and Technicians & AWS &  Spark Analytics, Heuristics/Rules \\
 %DIF >    J & Predictive maintenance (Non-NN) & Biochemicals & Sensor & Azure Cloud & - \\
%DIF >     K & Anomaly detection (Non-NN) & Real Estate & Meters & Azure Cloud & Scikit-learn, %XGBoost \\
    %DIF > L & Data analysis (Non-NN) & Pharmaceutical & Device,  Genome & Azure Cloud & R \\
    %DIF > M & Report/Document classification (Non-NN) & Healthcare & Internal systems & Azure Cloud & PyTorch, Scikit-learn (Classification) \\
%DIF >     N & Chatbots, Profiling (NN) & Finance & Internal systems & AWS & Watson(IBM), Tensorflow \\
%DIF >     O & ML pipeline automation (Non-NN) & IT services & Internal Systems & Azure Cloud & - \\
%DIF >     P & Marketing/campaign management (NN) & Media & Internal systems & AWS & Scikit-learn, Tensorflow, fastText \\
%DIF >     \hline
    \DIFaddbegin 


%DIF >     \end{tabular}%
%DIF >   \label{tab:data_source_storage_mlframeworks}
%DIF > \end{table*}% Add a role with multi column to add the text "caption"


%DIF > %%% with interviewees
%DIF > % Table summary of data collection, storage and ML frameworks
\DIFaddend \begin{table*}[t]
 \centering
  \caption{Summary of ML usecase, frameworks, data sources and storage platforms across cases. (*of the ML usecase)}
  \DIFdelbeginFL %DIFDELCMD < \begin{tabular}{p{0,5cm}p{3,4cm}p{2cm}p{2,7cm}p{1,6cm}p{2,7cm}}
%DIFDELCMD <     %%%
\DIFdelendFL \DIFaddbeginFL \begin{tabular}{p{0,5cm}p{2cm}p{2,8cm}p{1,8cm}p{2cm}p{1cm}p{2,3cm}}
    \DIFaddendFL \toprule
    \textbf{Case} & \DIFaddbeginFL \textbf{\DIFaddFL{Interviewee Role}} & \DIFaddendFL \textbf{ML usecase (Type)} & \textbf{Domain*} &\textbf{Data Source} & \textbf{Storage} & \textbf{ML-Framework} \\
    \toprule \\
    A & \DIFaddbeginFL & \DIFaddendFL Object Detection (NN) & IoT & Camera & \DIFdelbeginFL \DIFdelFL{Google Cloud }\DIFdelendFL \DIFaddbeginFL \DIFaddFL{GCP }\DIFaddendFL & Tensorflow \\
    B & \DIFaddbeginFL & \DIFaddendFL Form data extraction  (NN) & Finance & Internal Systems & AWS & Tensorflow \\
    C & \DIFaddbeginFL & \DIFaddendFL Form data extraction (NN) & Public Services & Internal Systems & On-premise & Tensorflow, PyTorch \\
    D & \DIFaddbeginFL & \DIFaddendFL Transcription (NN) & Healthcare & Internal Systems & \DIFdelbeginFL \DIFdelFL{Google Cloud }\DIFdelendFL \DIFaddbeginFL \DIFaddFL{GCP }\DIFaddendFL & Kaldi ASR framework \\
    E & \DIFaddbeginFL & \DIFaddendFL Speech based UI (NN) & E-Commerce & Open-source, End-users & \DIFdelbeginFL \DIFdelFL{Google Cloud }\DIFdelendFL \DIFaddbeginFL \DIFaddFL{GCP }\DIFaddendFL & PyTorch \\
   F & \DIFaddbeginFL & \DIFaddendFL MLOps (NN, Non-NN) & IT Services & Camera  & - & Multiple frameworks \\
    G & \DIFaddbeginFL & \DIFaddendFL MLOps (NN, Non-NN)  & Energy & Internal systems & AWS & Tensorflow, Scikit-Learn \\
    H & \DIFaddbeginFL & \DIFaddendFL Risk management (Non-NN) & Finance & Internal systems & AWS & Scikit-Learn, Heuristics \\
    I & \DIFaddbeginFL & \DIFaddendFL Predictive maintenance (Non-NN) & Engineering & Sensor and Technicians & AWS &  Spark Analytics, Heuristics/Rules \\
    J & \DIFaddbeginFL & \DIFaddendFL Predictive maintenance (Non-NN) & Biochemicals & Sensor & \DIFdelbeginFL \DIFdelFL{Azure Cloud }\DIFdelendFL \DIFaddbeginFL \DIFaddFL{AC }\DIFaddendFL & - \\
    K & \DIFaddbeginFL & \DIFaddendFL Anomaly detection (Non-NN) & Real Estate & Meters & \DIFdelbeginFL \DIFdelFL{Azure Cloud }\DIFdelendFL \DIFaddbeginFL \DIFaddFL{AC }\DIFaddendFL & Scikit-learn, XGBoost \\
    L & \DIFaddbeginFL & \DIFaddendFL Data analysis (Non-NN) & Pharmaceutical & Device,  Genome & \DIFdelbeginFL \DIFdelFL{Azure Cloud }\DIFdelendFL \DIFaddbeginFL \DIFaddFL{AC }\DIFaddendFL & R \\
    M & \DIFaddbeginFL & \DIFaddendFL Report/Document classification (Non-NN) & Healthcare & Internal systems & \DIFdelbeginFL \DIFdelFL{Azure Cloud }\DIFdelendFL \DIFaddbeginFL \DIFaddFL{AC }\DIFaddendFL & PyTorch, Scikit-learn (Classification) \\
    N & \DIFaddbeginFL & \DIFaddendFL Chatbots, Profiling (NN) & Finance & Internal systems & AWS & Watson(IBM), Tensorflow \\
    O & \DIFaddbeginFL & \DIFaddendFL ML pipeline automation (Non-NN) & IT services & Internal Systems & \DIFdelbeginFL \DIFdelFL{Azure Cloud }\DIFdelendFL \DIFaddbeginFL \DIFaddFL{AC }\DIFaddendFL & - \\
    P &\DIFaddbeginFL & \DIFaddendFL Marketing/campaign management (NN) & Media & Internal systems & AWS & Scikit-learn, Tensorflow, fastText \\
    \hline
    \DIFaddbeginFL 

       
    \multicolumn{7}{l}{
    Organizations often use cloud storage providers with data centres  either in Finland, close proximity to Finland or within}\\

    \multicolumn{7}{l}{the European Union area following customer preferences or regulatory constraints. In certain cases, data is stored in private }\\

    \DIFaddendFL \end{tabular}%
  \DIFdelbeginFL %DIFDELCMD < \label{tab:data_source_storage_mlframeworks}%%%
\DIFdelendFL \DIFaddbeginFL \label{tab:data_source_storage_mlframeworks_interviewees}\DIFaddendFL %
\end{table*}%
This section presents common practices observed across the cases as summarized in Table~\ref{tab:practices_challenges}.

\subsection{Data management}

\underline{\emph{Data collection and storage}}
% Batch vs Real-time streams
are handled in various ways: batch loading data from internal systems, streaming from devices/sensors, extracting from third-party vendors APIs or open-source repositories. The training data is then commonly stored in cloud platforms, as shown from data sources in Table ~\ref{tab:data_source_storage_mlframeworks_interviewees}.

Low-level metrics such as IOPS (I/O Operations Per Second) are considered when choosing a storage architecture; data fetching can constitute a sizeable amount of the overall model training time. Case E uses a mounted discs solution instead of a network drive accessed via a web interface. %I/O bound process.

\underline{\emph{Data storage formats}} are factored in when considering the scalability of data processing pipelines, data portability between computing environments, and support of different ML frameworks. Case H uses Apache Parquet in favor of CSV (Comma Separated Values) or TSV (Tab Separated Values) file formats commonly used to store structured data for analytics purposes. Case G uses NetCDF to implement a generic data interface to abstract data across ML frameworks and computing platforms. 

% \subsubsection{Data versioning}

\underline{\emph{Data discoverability and accessibility}}
%affects the rate of experimentation and development of ML features. This 
is emphasized in setups that feature a data lake or where different data types are collected. Case O describes a solution to this problem based on maintaining a \textit{data catalog} where data and its value are described. %This supports faster identification of data and its associated value. An alternative approach is to assign metadata to ingested data and the metadata can be made visible through custom tools.
Data governance and related processes can limit the use and scope of data accessible for ML purposes.
%An example of such a data migration pattern involves an organization streaming data from its IoT devices to it's own cloud storage then a subset of the data is sent to a third party entity processor for cleaning or model training over an agreed interval. %https://www.linkedin.com/pulse/top-10-data-governance-anti-patterns-analytics-dave-wentzel

\underline{\emph{Data validation}}techniques are commonly applied as a means of controlling data quality. However, data types influence the type of validation approaches used. Validation of Image/video, speech, and text tend to require human actors supported by custom tools. For example, a human validator ensures that objects fall within the annotated bounding boxes in an object detection setting. While a human speech validator ensures recorded utterances are coherent and consistent with corresponding text. Case D uses additional heuristics for detecting anomalies between generated texts and submitted utterances. Numerical data types usually are easier to validate automatically.

Data validation in Case O is done at a schema and data level. A dedicated data stewards maintain the schema. Delegating quality control ensures a team managing the data lake ingests data indiscriminately. When data is sourced from third-party vendors, the vendor is expected to maintain quality controls (Case P).


\underline{\emph{Data integrity}}
controls ensure data is not changed unexpectedly. Case D and F apply hashing as part of data processing pipelines; this ensures training data is verifiable and traceable with respect to a model's lineage. Additionally, this practice ensures that attempts to overwrite data are flagged appropriately.

Generally, when hashing is not a suitable approach, for example, when dealing with image files, other custom tooling and heuristics are used to perform anomaly detection; Cases B and I make use of this approach.

\underline{\emph{Data labelling and annotation}} tend to be undertaken manually using custom tools developed to standardize the process. Inconsistent labels are sometimes encountered due to subjective interpretations, resulting in poor data quality. Case B implements a standardized way of normalizing and giving common meaning to concepts to overcome such issues. 

\underline{\emph{Data understanding}} requires domain knowledge for teams to generate valuable insights from data in specialized domains. Domain knowledge is cited as a necessity in the entire life cycle of the data. For example, handling data from chemical processes or mechanical parts of large systems is represented in cases I, J, and L.

In general, challenges in data management practices are mainly attributed to data quality aspects. For example, sensor problems, inconsistent labeling, programming errors in data handling software, etc.



