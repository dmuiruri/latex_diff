Today, artificial intelligence (AI) is incorporated in many real-world software systems and services. However, research on the development, deployment, and maintenance of AI-enabled systems in industrial settings report this to be a challenging task \cite{Sculley2015, Lwakatare2019}. Large companies, like Google \cite{Baylor2017} and Facebook \cite{Hazelwood2018Facebook}, often report their development practices and infrastructure for AI solutions that are useful for learning. However, many organizations are yet to adopt and tailor the suggested development practices and infrastructures to narrow the gap from mere prototyping to deploying to production AI solutions \cite{Serban2020Practices}. 

Machine learning (ML) is a subset of AI, and its techniques involve the use of high-quality data. ML logic is not explicitly programmed but is rather learned from data. The development of industrial ML-enabled software systems involves ML pipelines that consist of several interlocking steps. To support the different steps, end-to-end in one environment, ML platforms like TensorFlow Extended (TFX) \cite{Baylor2017} have been proposed to ensure increased automation across the steps.

Since industrial ML pipelines can be complex, understanding their characteristics is essential. In a large organization like Google, some 3000 ML pipelines comprising over 450,000 trained ML models continuously update the models at least seven times a day  \cite{Doris2021MLPipelines}. The need to support regular model training and updates in production is a common requirement in most industrial ML-enabled systems because the performance of models deteriorates over time  \cite{Sculley2015}.

Most empirical literature presents development and maintenance practices of ML-enabled systems from the perspective of a single, often large and experienced online organization. In contrast, we aim to provide empirical evidence of the practices and infrastructure setups across a diverse set of companies in various domains. Through interviews, this study investigated ML workflow practices and toolchains that are used in the development, deployment, and maintenance of ML-enabled systems in selected organizations in Finland. Our main contributions include:
\begin{itemize}
    \item Empirical evidence of the enacted practices in ML workflows  (Section \ref{sec:practices}),
    \item Tool adoption in ML pipelines (Section \ref{sec:tools}) and areas of future research (Section \ref{sec:discussion}).
    %\item %Remaining challenges (Section \ref{sec:challenges}) and 
\end{itemize}
